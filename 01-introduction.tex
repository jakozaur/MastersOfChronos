\chapter{Introduction}

The Internet is one of the most successful technologies in the humankind history.
This is party due to its solid foundations --- the protocols, designed and carefully optimized for a high throughput, low latencies and outstanding reliability.
As a result of this architectural decisions, devices connected by the Internet can communicate efficiently.

However, novel ideas on the evolution of the Internet require revisiting the design goals of the protocols.
In particular, power consumption was not considered as a high-priority objective.
Consequently, today's Internet devices either must be permanently connected to a power grid or live on batteries for only a few hours.

% Internet of things
\section{Internet of Things}

The lack of focus on power constitutes a major problem for the novel vision of the global network: the so-called Internet of Things\footnote{term first used by Kevin Ashton \cite{InternetOfThings}}.
The principal idea behind this vision is making ordinary physical objects --- things --- Internet-enabled, so that they can communicate with each other without a human in the loop.
Connecting ordinary things to the Internet would open a plethora of possibilities in information technology, including tracking items (e.g. monitoring art images \cite{GuArtNet}), reducing resource waste (e.g. controlling city lights \cite{singhvi2005intelligent}) and helping us better understand our environment (e.g. monitoring wildlife \cite{liu2009long}).
A major challenge is that the most of things are currently not connected to a power grid.
What is more, in many cases, employing power cables would preclude normal use of the things.
Likewise, many things cannot be expected to be maintained regularly, for example, to replace their batteries.
Therefore, the Internet of Things requires a rethought approach.

% Philosophy

More specifically, a technology used to connect everyday things to the Internet has to meet the following core requirements:
\begin{itemize}
  \item \textbf{Fully wireless.} Cables are not feasible in most use cases. Therefore, both power and communication should not require them.
Data should be transmitted using radio, infrared or sonic waves.
  \item \textbf{Low-power.} Energy is a precious resource for a wireless node.
It is needed by all of the node's components and can be provided from batteries or the environment (by harvesting).
Energy harvesting opportunities are limited without increasing the node's complexity or requiring a specific environment (e.g. a sunny place).
Therefore, batteries are currently the main source of energy.
However, their capacity will probably increase slowly.
Consequently, minimizing energy usage seems to be the only reasonable solution.
  \item \textbf{Inexpensive hardware.} Each node placed in a thing should cost only a small percentage of the thing itself.
Ideally it should cost at least an order of magnitude less than current generation sensor nodes.
  % \item low duty cycle - some power savings are constrained due to physics limitation. Most notable wireless communication. So using most energy consuming components for only a tiny percent of total time seems to be the best option. That way even current sensors are capable of working even years without a service.
\end{itemize}

% Problem
\section{Routing}
Moore's law brings hope to produce better hardware: while keeping the same performance we are able to produce less power hungry nodes every year.
However, software and protocols need to be developed to support this new technology.

One of the fundamental problems in sensor networks is routing.
The goal of a routing protocol is finding paths in the network along which nodes can communicate.
In a typical environment a single node can communicate with only a few nearest nodes.
Sending a packet to another node thus requires passing it by intermediate nodes along the path provided by the routing protocol.
Doing this efficiently, reliably and with reasonable latencies is a major challenge.

% Mobility
What makes this problem even more interesting is the fact that things often move in the real world and so would nodes embedded into them.
Mobility brings additional problems to routing protocols.
The network topology changes over time and algorithms have to ensure reliable packet delivery without causing too much redundant communication.


\section{Problem statement}
The ``Scalable Self-Managed Point-to-Point Routing for the Internet of Things Applications'' (rhoRoute) project aims to investigate the problem of routing and developing better routing algorithms for mobile networks.
In order to accomplish this goal, an appropriate infrastructure is required.
Often algorithms that work well in simulation do not necessarily achieve similar results in the real world.
Moreover, the technical implementation of a routing algorithm is usually non-trivial, because of node hardware constraints.

Real sensor node hardware is vital equipment required for evaluations.
An ideal mobile sensor node should provide a good radio range, different sensors, an LCD display and buttons.
For this purpose, wireless smart watches constitute a promising platform for testing mobile routing protocols.
The model Chronos eZ430\footnote{\cite{eZ430Chronos}} was chosen, because of its decent specification, popularity and price.

However, as we argue above, being able to develop software for the hardware is equally important.
Routing algorithms are often more complicated than the manufacturer's suggested use cases (monitoring sport performance). 
In order to be of a high quality software, the system should consist of modular, reusable components.
Researchers commonly use TinyOS\footnote{\cite{TinyOS}} to achieve this goal.
Moreover, TinyOS also provides many useful features and increases programmers' productivity.

Unfortunately, eZ430 Chronos is not supported by TinyOS.
The problem this Master's thesis addresses is porting TinyOS to eZ430 Chronos and providing an appropriate programming environment. 

% What is also very important, by using the same operating system (OS) --- TinyOS, researches are able to compare their algorithms and evaluation more accurately. 
% Common OS also simplify cross-devices interoperability, in particular eZ430 with GNode.
% This was another requirement for project rhoRoute, because the existing testbed\footnote{located at Faculty of Mathematics, Informatics and Mechanics, University of Warsaw} uses GNode sensor nodes.
% Accomplishing this requirement allows testing hybrid static-mobile networks.
Cross-device interoperability is also an important issue.
Port of TinyOS to eZ430 simplify radio communication with other devices which also use the same OS.
This requirement is motivated by the need for testing protocols in hybrid networks combining G-Nodes and eZ430, in which some nodes are static whereas others are mobile.

% Contributions
\section{Contributions}
Our contributions are manifold:
\begin{itemize}
  \item A new TinyOS platform for eZ430 Chronos.
It is a port of the existing operating system into the new hardware architecture.
It consist of device drivers for timers, keyboard, radio, accelerometer, altitude meter, beeper and LCD screen.
The drivers provide a high-level abstraction and encapsulate hardware details.
  \item A programming environment for eZ430. It is set of configured and adapted developer tools. They enable and simplify programming TinyOS applications on eZ430.
The environment consists of mspdebug (allows deploying programs on a device and debugger), different variants of printf, Eclipse (Integrated Development Environment), vim (a text editor) and a virtual machine image.
  \item Experiments investigating power usage and radio connectivity of eZ430.
We present both, the methodology of experiments we conducted and their results.
This data extend our knowledge on the Chronos devices beyond official specification.
  \item Application feasibility studies.
We analyze what the possible use cases are for Chronos.
The study is based on experiments and our experience.
It provides examples of future applications and how suitable they are for the Chronos platform.
\end{itemize}

Finally, It is worth mentioning, that this very thesis is a major contribution itself. More specifically, it constitutes an introduction to programming in TinyOS on eZ430. Therefore, it is likely to be useful not only for future members of the rhoRoute project, but also possibly for researches and developers from other groups.

% Thesis organization
\section{Thesis organization}
TODO: rewrite when thesis is done.

The thesis is organized into 8 chapters. Chapter 2 discusses the Chronos hardware technical capabilities. Chapter 3 describes programming environment which was assembled to write drivers more efficiently and help application developers. Chapter 4 discusses TinyOS and NesC architecture, while chapter 5 describes in detail implementation work on device drivers. Chapter 6 is about all of the experiments and evaluations on Chronos. The further application for chronos are identified and analyzed in chapter 7. The last chapter (8) summarize the progress and points further areas for improvement.
