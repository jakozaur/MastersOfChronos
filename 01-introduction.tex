\chapter{Introduction}

The Internet is one of the most successful technologies in the humankind history. This is party due to its solid fundamentals - the protocols, which were designed and carefully optimised for high throughput, low latencies and outstanding reliability. As a result of this architectural decisions, devices connected by the Internet can communicate efficiently.

However, novel ideas on the evolution of the Internet require revisiting the design goals of the protocols. In particular, power consumption was not considerate as an prioritised objective. Consequently, today's Internet devices either must be permanently connected to a power grid or live only a few hours on batteries.

% Internet of things
\section{Internet of Things}

The lack of focus on power constitutes a major problem for the novel vision of the global network: the so-called Internet of Things (\footnote{term first used by Kevin Ashton in RFID Journal, 22 July 2012: \url{http://www.rfidjournal.com/article/view/4986}}). The principal idea behind this vision is making ordinary physical objects - things - Internet-enabled, so they can communicate each other without human in the loop. Connecting ordinary things to the Internet would open plethora of new possibilities in information technology. Including tracking items (e.g. monitoring art images), reducing resource waste (e.g. controlling city lights) and would help us to better understand our environment (e.g. monitoring wildlife \cite{liu2009long}).
The major challenge is that majority of things are currently not connected to a power grid. What is more, in many cases, employing power cables preclude normal use of things. Likewise, many things cannot be expected to be maintained regularly, for example, to swap the batteries.
Therefore, the Internet of Things requires a rethought approach to the problem.

% Philosophy

More specifically, a technology used to connect everyday things to the Internet has to meet following core requirements:
\begin{itemize}
  \item fully wireless - cables are not feasible in most use cases. Therefore, both power and communication should not required them. Data could be transmitted using radio, infrared, light or sonic waves.
  \item low-power - the energy is the precious resource in wireless nodes. It is needed by all of the components and could be provided from batteries or harvesting energy. The harvesting opportunities are very limited without hugely increasing the device complexity or requiring very specific environment (e.g. sunny place). The batteries are currently the main source and their capacity will probably increase slowly over the next years. So optimising for low-power usage seems to be only viable solution.
  \item inexpensive hardware - each node placed in a thing should cost only a small percentage of the thing itself. Ideally they should cost at least an order of magnitude less then current generation. E.g. we might want to place microcontroler in every LED light bulb.
  % \item low duty cycle - some power savings are constrained due to physics limitation. Most notable wireless communication. So using most energy consuming components for only a tiny percent of total time seems to be the best option. That way even current sensors are capable of working even years without a service.
\end{itemize}

% Problem
\section{Routing}
Moore's law brings hope to produce better hardware. While keeping the same performance we are able to produce less power hungry nodes every year. However, software and protocols need to be developed to support this new technology.

One of the fundamental problems in sensor networks is routing. The goal of routing protocol is finding paths in the network along which nodes can communicate.
In a typical environment a single node can only communicate with a few nearest nodes. Sending a packet to another node requires passing it by intermediate nodes along the path provided by the routing protocol. Doing this efficiently, reliably and with reasonable latencies is a major challenge.

% Mobility
What makes this problem even more interesting, is the fact that the things often moves in real world, so do nodes.
Mobility bring additional problems to routing protocols.
The structure changes over time and algorithms have to be at the same time reliability without causing too much redundant communication.

The rhoRoute project aims to investigate the problem and develop better routing algorithms for mobile networks. In order to accomplish this goal, an appropriate infrastructure is required.
Often algorithms which work well in simulation do not necessarily achieve similar results in the real world. Moreover the technical implementation of a routing algorithm is usually non-trivial, because of node hardware constraints.

\section{Problem statement}
In order to simplify programming model and be able to compare routing algorithms sensor network researches utilise TinyOS operating system. It provides many useful features and increase programmer productivity. However, it is not supported by many devices, including Chronos eZ430. The aim of this thesis is to port TinyOS to eZ430. It requires assembling a better programming environment and writing the device drivers.

In addition to that, by using TinyOS on the mobile node eZ430 will be inter-operable with gnode. The gnode are a static nodes which were used to construct a static network of sensor nodes at MIMUW faculty (called testbed). This allow evaluation of routing algorithm using static or/and mobile nodes. 

% Contributions
\section{Contributions}
The goal of this thesis project is to provide:
\begin{itemize}
  \item Platform Chronos in TinyOS - port of TinyOS on eZ430. Consist mainly of device drivers implementation.
  \item Programming environment - set of configured and adapted developer tools. They enable and simplify programming TinyOS applications of eZ430.
  \item Conduct experiments of power usage and radio connectivity - methodology of experiments and their results. Their extend our knowledge about device beyond official specification.
  \item Feasibility studies. Analyse what are the possible use case for Chronos. It is based on experiments and our experience. The study provides examples of future applications and how suitable are they to Chronos platform.
  \item Thesis paper. The paper itself could be an useful introduction to programming in TinyOS on eZ430.
\end{itemize}

% Thesis organisation
\section{Thesis organisation}
TODO: rewrite when thesis is done.

The thesis is organised into 8 chapters. Chapter 2 discusses the Chronos hardware technical capabilities. Chapter 3 describes programming environment which was assembled to write drivers more efficiently and help application developers. Chapter 4 discusses TinyOS and NesC architecture, while chapter 5 describes in detail implementation work on device drivers. Chapter 6 is about all of the experiments and evaluations on Chronos. The further application for chronos are identified and analysed in chapter 7. The last chapter (8) summarise the progress and points further areas for improvement.
