% TODO: example: change
\begin{thebibliography}{99}
\addcontentsline{toc}{chapter}{Bibliography}

\bibitem[BerkDB]{berkdb}
M. A. Olson, K. Bostic and M. Seltzer,
\textit{Berkeley DB},
In USENIX (June 1999), pages 183-192

\bibitem[Liu2009]{liu2009long}
Liu, N.H. and Wu, C.A. and Hsieh, S.J.,
\textit{Long-Term Animal Observation by Wireless Sensor Networks with Sound Recognition},
Wireless Algorithms, Systems, and Applications (2009)

%\bibitem[Bea65]{beaman} Juliusz Beaman, \textit{Morbidity of the Jolly
%    function}, Mathematica Absurdica, 117 (1965) 338--9.
%
%\bibitem[Blar16]{eb1} Elizjusz Blarbarucki, \textit{O pewnych
%    aspektach pewnych aspektów}, Astrolog Polski, Zeszyt 16, Warszawa
%  1916.
%
%\bibitem[Fif00]{ffgg} Filigran Fifak, Gizbert Gryzogrzechotalski,
%  \textit{O blabalii fetorycznej}, Materiały Konferencji Euroblabal
%  2000.
%
%\bibitem[Fif01]{ff-sr} Filigran Fifak, \textit{O fetorach
%    $\sigma$-$\rho$}, Acta Fetorica, 2001.
%
%\bibitem[Głomb04]{grglo} Gryzybór Głombaski, \textit{Parazytonikacja
%    blabiczna fetorów --- nowa teoria wszystkiego}, Warszawa 1904.
%
%\bibitem[Hopp96]{hopp} Claude Hopper, \textit{On some $\Pi$-hedral
%    surfaces in quasi-quasi space}, Omnius University Press, 1996.
%
%\bibitem[Leuk00]{leuk} Lechoslav Leukocyt, \textit{Oval mappings ab ovo},
%  Materiały Białostockiej Konferencji Hodowców Drobiu, 2000.
%
%\bibitem[Rozk93]{JR} Josip A.~Rozkosza, \textit{O pewnych własnościach
%    pewnych funkcji}, Północnopomorski Dziennik Matematyczny 63491
%  (1993).
%
%\bibitem[Spy59]{spyrpt} Mrowclaw Spyrpt, \textit{A matrix is a matrix
%    is a matrix}, Mat. Zburp., 91 (1959) 28--35.
%
%\bibitem[Sri64]{srinis} Rajagopalachari Sriniswamiramanathan,
%  \textit{Some expansions on the Flausgloten Theorem on locally
%    congested lutches}, J. Math.  Soc., North Bombay, 13 (1964) 72--6.
%
%\bibitem[Whi25]{russell} Alfred N. Whitehead, Bertrand Russell,
%  \textit{Principia Mathematica}, Cambridge University Press, 1925.
%
%\bibitem[Zen69]{heu} Zenon Zenon, \textit{Użyteczne heurystyki
%    w~blabalizie}, Młody Technik, nr~11, 1969.

\end{thebibliography}
