% TODO: example: change
\begin{thebibliography}{99}
\addcontentsline{toc}{chapter}{Bibliography}

\bibitem[BerkDB]{berkdb}
M. A. Olson, K. Bostic and M. Seltzer,
\textit{Berkeley DB},
In USENIX (June 1999), pages 183-192

\bibitem[Liu2009]{liu2009long}
Liu, N.H. and Wu, C.A. and Hsieh, S.J.,
\textit{Long-Term Animal Observation by Wireless Sensor Networks with Sound Recognition},
Wireless Algorithms, Systems, and Applications (2009)

\bibitem[InternetOfThings]{InternetOfThings}
Kevin Ashton,
\textit{That 'Internet of Things' Thing},
RFID Journal (July 2012) \url{http://www.rfidjournal.com/article/view/4986}

\bibitem[GuArtNet]{GuArtNet}
GuArtNet whitepaper \url{http://www.sownet.nl/download/GuArtNet\_Whitepaper\_English.pdf}

\bibitem[singhvi2005intelligent]{singhvi2005intelligent}
Singhvi, V. and Krause, A. and Guestrin, C. and Garrett Jr, J.H. and Matthews, H.S.,
\textit{Intelligent light control using sensor networks},
Proceedings of the 3rd international conference on Embedded networked sensor systems (2005), pages 218-229

\bibitem[NesC]{NesC}
  David Gay, Philip Levis, and Robert von Behren.
  \textit{The nesC Language: A Holistic Approach
  to Networked Embedded Systems}.  \newblock In
  {\em Proceedings of the ACM SIGPLAN 2003
  Conference on Programming Language Design and
  Implementation (PLDI)}, 2003.

\bibitem[TOSMock]{TOSMock}
  Piotr Glazar, \textit{A unit-testing framework
  for wireless sensor networks.} \\ Master thesis,
  University of Warsaw (2012)

\bibitem[TOSSIM]{TOSSIM}
  P.~Levis, N.~Lee, M.~Welsh, and D.~Culler.
  \textit{Tossim: accurate and scalable
  simulation of entire tinyos applications.}
  \newblock In {\em SenSys '03: Proceedings of the
  1st international conference on Embedded
  networked sensor systems}, pages 126--137, New
  York, NY, USA, 2003. ACM.

\bibitem[TinyOS]{TinyOS}
Levis, P. and Madden, S. and Polastre, J. and Szewczyk, R. and Whitehouse, K. and Woo, A. and Gay, D. and Hill, J. and Welsh, M. and Brewer, E. and others,
\textit{TinyOS: An operating system for sensor networks}
Ambient intelligence (2005)

\bibitem[TOSProg]{TOSProg}
  Philip Levis, \textit{TinyOS Programming}, October 27, 2006 \\
  \url{http://www.tinyos.net/tinyos-2.x/doc/pdf/tinyos-programming.pdf}

\bibitem[eZ430Chronos]{eZ430Chronos}
  eZ430-Chronos User Guide \\
  \url{http://www.ti.com/lit/ug/slau292c/slau292c.pdf}

\bibitem[CC430ds]{CC430ds}
  CC430 family datasheet\\
  \url{http://www.ti.com/general/docs/lit/getliterature.tsp?literatureNumber=swru191c&fileType=pdf}

\bibitem[CC430F6137ds]{CC430F6137ds}
  CC430F6137 MCU datasheet \\
  \url{http://www.ti.com/general/docs/lit/getliterature.tsp?genericPartNumber=cc430f6137&fileType=pdf}

\bibitem[UML2ForTOS]{UML2ForTOS}
  Sebastian A. Bachmaier
  \textit{UML 2.0 for modeling TinyOS components} \\
  \url{http://www.ti5.tu-harburg.de/events/fgsn09/proceedings/fgsn_087.pdf}

\bibitem[TOSnet]{TOSnet}
  TinyOS website \\
  \url{http://www.tinyos.net}

\bibitem[TOSw]{TOSw}
  TinyOS topic on Wikipedia \\
  \url{http://en.wikipedia.org/wiki/TinyOS}
%\bibitem[Bea65]{beaman} Juliusz Beaman, \textit{Morbidity of the Jolly
%    function}, Mathematica Absurdica, 117 (1965) 338--9.
%
%\bibitem[Blar16]{eb1} Elizjusz Blarbarucki, \textit{O pewnych
%    aspektach pewnych aspektów}, Astrolog Polski, Zeszyt 16, Warszawa
%  1916.
%
%\bibitem[Fif00]{ffgg} Filigran Fifak, Gizbert Gryzogrzechotalski,
%  \textit{O blabalii fetorycznej}, Materiały Konferencji Euroblabal
%  2000.
%
%\bibitem[Fif01]{ff-sr} Filigran Fifak, \textit{O fetorach
%    $\sigma$-$\rho$}, Acta Fetorica, 2001.
%
%\bibitem[Głomb04]{grglo} Gryzybór Głombaski, \textit{Parazytonikacja
%    blabiczna fetorów --- nowa teoria wszystkiego}, Warszawa 1904.
%
%\bibitem[Hopp96]{hopp} Claude Hopper, \textit{On some $\Pi$-hedral
%    surfaces in quasi-quasi space}, Omnius University Press, 1996.
%
%\bibitem[Leuk00]{leuk} Lechoslav Leukocyt, \textit{Oval mappings ab ovo},
%  Materiały Białostockiej Konferencji Hodowców Drobiu, 2000.
%
%\bibitem[Rozk93]{JR} Josip A.~Rozkosza, \textit{O pewnych własnościach
%    pewnych funkcji}, Północnopomorski Dziennik Matematyczny 63491
%  (1993).
%
%\bibitem[Spy59]{spyrpt} Mrowclaw Spyrpt, \textit{A matrix is a matrix
%    is a matrix}, Mat. Zburp., 91 (1959) 28--35.
%
%\bibitem[Sri64]{srinis} Rajagopalachari Sriniswamiramanathan,
%  \textit{Some expansions on the Flausgloten Theorem on locally
%    congested lutches}, J. Math.  Soc., North Bombay, 13 (1964) 72--6.
%
%\bibitem[Whi25]{russell} Alfred N. Whitehead, Bertrand Russell,
%  \textit{Principia Mathematica}, Cambridge University Press, 1925.
%
%\bibitem[Zen69]{heu} Zenon Zenon, \textit{Użyteczne heurystyki
%    w~blabalizie}, Młody Technik, nr~11, 1969.

\end{thebibliography}
