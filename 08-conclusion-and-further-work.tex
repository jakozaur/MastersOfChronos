\chapter{Conclusions and further work}

\section{Conclusions}

% Complete environment

The aim of this project was to provide infrastructure to evaluate routing algorithms in mobile sensor networks.
It was accomplished by porting TinyOS to eZ430 Chronos, which is the first mobile node supported by this operating system.

Our work start from scratch, without essential programming tools.
Reusing existing software, we assembled convenient development environment, which consist of compiler, debugger, chip programmer, printf library, build system, IDE and virtual machine.
The core part of our work was implementation of TinyOS device drivers for eZ430 in nesC programming language.
We also performed several evaluation in order to test radio power consumption, range and throughput.
At last, but not at least, we provided example application --- Zordon for Chronos, which demonstrate its hardware features..

Although the project may seem to be limited to low-level programming, it delivered a robust, well-designed abstraction, which encapsulate eZ430 technical details.
It is a radical improvement over manufacturer's tools, which require handling manually hardware issues and are suitable only for simple applications.
Our programming environment supports modular, object-oriented, cross-platform applications.
In addition to that, it is possible to take advantage of TineOS ecosystem:
ready to use applications as well as tools and libraries (e.g. unit-test framework \cite{TOSMOCK}).

We believe it is the great leap forward in development of applications for mobile sensor networks.

% complete programming environment

\section{Further work}
There are a lot of research projects which could take advantage of our TinyOS port to eZ430.
Especially two of them seems to be a next step.

\subsection{Evaluations of existing routing protocols}
There are many published articles about sensor network protocols which were implemented in TinyOS.
However, there was no mobile hardware nodes to evaluate them in non-static environment.
Using our TinyOS port to eZ430 it is much easier to perform that kind of experiments and provide valuable results for the research community.

\subsection{Better low-power listening}
Low-power consumption is one of the key requirements for sensor nodes.
Radio communication usually use most of the energy, so it is the best candidate for further optimizations.
The savings are achieved by turning off radio module whenever it is possible.
However, deciding when radio listening is needed is a hard problem.
The eZ430 Chronos has implementation of low-power listening, but it is based on simple algorithm.

Various scientific papers presents more sophisticated solutions \cite{BMAC} and \cite{DCCLPL}.
Their power performance is claim hardware and application specific.
Although, implementation and evaluation of those algorithms would require significant amount of work, it is likely that they would be substantial better implementation of low-power listening.


% we need a better MAC layer
