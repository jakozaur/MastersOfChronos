\chapter{Conclusions and future work}

\section{Conclusions}

% Complete environment

The aim of this project was to provide an infrastructure for evaluating routing algorithms in mobile sensor networks.
It was accomplished by porting TinyOS to eZ430 Chronos, which is, to the best of our knowledge, the first mobile watch supported by this operating system.

Our work started from scratch, without essential programming tools.
Reusing existing software, we assembled a convenient development environment, which consist of a compiler, debugger, chip programmer, printf library, build system, IDE and virtual machine.
However, The core part of our work was the implementation of TinyOS device drivers for eZ430.
We also performed a preliminary evaluation in order to test the radio power consumption, range and throughput.
Last, but not least, we provided a sample application, Zordon, which demonstrates eZ430 hardware features.

Although the project may seem limited to low-level programming, it delivered a robust, well-designed abstractions, which encapsulate eZ430 technical details.
It is a radical improvement over the manufacturer's tools, which require manually handling hardware issues and are suitable only for simple applications.
Our programming environment supports modular, object-oriented, cross-platform applications.
In addition to that, it is possible to take advantage of the TineOS ecosystem:
ready to use applications as well as tools and libraries (e.g. an unit-test framework \cite{TOSMock}).

We believe it is a great leap forward in development of applications for mobile sensor networks.

% complete programming environment

\section{Future work}
There are a lot of research projects which could take advantage of our TinyOS port to eZ430.
Especially two of them seem to be ideal as a next step.

\subsection{Evaluations of existing routing protocols}
There are many published articles about sensor network protocols which were implemented in TinyOS.
However, there were few mobile hardware nodes to evaluate them in non-static environment.
Using our TinyOS port to eZ430 it is much easier to perform that kind of experiments and provide valuable results for the research community.

\subsection{Better low-power listening}
Low power consumption is one of the key requirements for sensor nodes.
Radio communication usually uses most energy, so it is the best candidate for further optimizations.
The savings are achieved by turning off the radio module whenever it is possible.
However, deciding when radio listening is needed is a hard problem.
Our eZ430 Chronos port has an implementation of low-power listening, but it is based on a simple algorithm.

Various scientific papers presents more sophisticated solutions (e.g. \cite{DCCLPL}).
Although implementation and evaluation of those algorithms would require a significant amount of work, it is likely that these solutions would be substantially better than current low-power listening.


% we need a better MAC layer
